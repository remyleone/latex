\chapter{Le mod�le}

\begin{chapintro}
  \malettrine{D}{ans} ce chapitre, nous pr�sentons le mod�le\dots
\end{chapintro}

%%% ----------------------------------------
%%% SECTION 1 --- CONSTRUCTION DU MOD�LE ---

\section{Construction du syst�me dynamique}

Dans cette  section, nous construisons  la classe de Processus d'int�r�t.
La  construction s'op�re par  morceaux successifs sur les trajectoires 
des processus.

Les   trajectoires  du  processus   peuvent  repr�senter,   selon  les
applications, l'�tat d'une  population de particules (en neutronique),
d'une population  de bact�ries  (en biologie), la  concentration d'une
prot�ine dans une  solution (en chimie) ou, dans  notre cas, l'�tat du
niveau  de d�gradation  d'une structure  (la taille  d'une  fissure se
propageant dans une structure).

Auparavant,  nous avons  besoin d'introduire  quelques  d�finitions et
notations  concernant  les  processus   de  Markov  et  les  processus
markoviens de saut.

\subsection{Rappels sur les processus de Markov}

Nous d�finissons un processus de Markov de la mani�re suivante :

\begin{definitionf}\label{def:processus_markov}
  Soit  $(\Omega,\mathcal{F},\mathbb{P})$  un  espace  probabilis� et  soit  $(X_t,t\in
  \mathbb{R}_+)$  un  processus al�atoire  �  valeurs  dans  un espace  d'�tat
  mesurable   $E$  de   tribu  $\mathcal{E}$.    Notons  $\mathcal{F}_t$   la  tribu
  d'�v�nements   engendr�e    par   $(X_s,0\leq   s    \leq   t)$   et
  $(\mathcal{F}_t)_{t\in\mathbb{R}_+}$ la filtration associ�e.

  Le processus  $X_t$ est  un {\em processus  de Markov} si  pour tout
  $B\in \mathcal{E}$ et pour  tout $s,t \in \mathbb{R}_+$ tels que $0  \leq s < t$,
  il satisfait
  \begin{equation*}
    \mathbb{P}(X_t \in B | \mathcal{F}_s) = \mathbb{P}(X_t \in B | X_s), \qquad p.s.
  \end{equation*}
  De plus, $X_t$ est {\em homog�ne  par rapport au temps} si pour tout
  $t,s\in \mathbb{R}_+$ et pour tout $x\in E$, alors
  \begin{equation}
    \label{eq:c2_def_homogene}
    \mathbb{P}(X_t\in B | X_0=x) = \mathbb{P}(X_{t+s}\in B | X_{s}=x).
  \end{equation}
  Pour  un processus  de Markov  homog�ne, nous  notons  $P(x,B,t)$ la
  probabilit� \eqref{eq:c2_def_homogene}.  La fonction d�finie par $P:
  (x,B,t) \rightarrow P(x,B,t)$ pour $x\in E, B\in\mathcal{E},t\in\mathbb{R}_+$ est
  appel�e \emph{fonction de transition} du processus.
\end{definitionf}

La   figure  \ref{fig:c2_markov}   repr�sente  une   trajectoire  d'un
processus markovien de saut, avec les notations associ�es.

\figScale{c2_markov}{Trajectoire type d'un processus de saut}

%%% --------------------------
%%% CONCLUSION DU CHAPITRE ---

\section*{Conclusion du chapitre}
\addcontentsline{toc}{section}{Conclusion}

Encore une\dots


